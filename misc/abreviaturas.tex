
\makeglossaries
\newglossaryentry{latex}
{
    name=latex,
    description={Is a mark up language specially suited 
    for scientific documents}
}
%%Falta describir mejor estas palabras
\newglossaryentry{HFW}{
    type=\acronymtype,
    name=HFW,
    description={Palabras de alta frecuencia},
    first={Palabras de alta frecuencia (HFW por sus siglas en inglés)},
}

\newglossaryentry{F-score}{
    type=\acronymtype,
    name=F-score,
    description={Es una medida de evaluación que correlaciona el recall y la precisión, también se conoce como F-measure, F1-score},
    first={F-score},
}

\newglossaryentry{CW}{
    type=\acronymtype,
    name=CW,
    description={Palabras de contenido},
    first={Palabras de contenido (CW por sus siglas en inglés)},
}

\newglossaryentry{DT}{
    type=\acronymtype,
    name=DT,
    description={Decision tree},
    first={Árbol de decisión (DT por sus siglas en inglés)},
}

\newglossaryentry{FT}{
    type=\acronymtype,
    name=FT,
    description={Functional trees},
    first={Functional Trees (FT)},
}

\newglossaryentry{RF}{
    type=\acronymtype,
    name=RF,
    description={Random forest},
    first={Random Forest (RF)},
}

%%Hasta aquí
\newglossaryentry{NB}{
    type=\acronymtype,
    name={NB},
    description={Naive Bayes},
    first={Clasificador bayesiano ingenuo (NB por sus siglas inglés)\glsadd{NBg}},
    see=[Glosario:]{NBg}
}

\newglossaryentry{tf-idfg}{
    name=tf-idf,
    description={Método de análisis de textos que propone un valor de importancia a las palabras que forman parte de un corpus},
    first={Frecuencia de Término – Frecuencia Inversa de Documento (comunmente conocido como Tf-Idf)},
}

\newglossaryentry{tf-idf}{
    type=\acronymtype,
    name={tf-idf},
    description={Frecuencia de Término – Frecuencia Inversa de Documento (comunmente conocido como Tf-Idf)},
    first={Frecuencia de Término – Frecuencia Inversa de Documento (comunmente conocido como Tf-Idf)},
    see=[Glosario:]{tf-idfg}
}

\newglossaryentry{n-grams}{
    name={n-grams},
    description={Es una forma de obtener el vocabulario con el cual se modelará un texto, n corresponde al número de palabras juntas serán tomadas en cuenta como una entidad única.},
    first={n-grams},
}

\newglossaryentry{MaxEnt}{
    type=\acronymtype,
    name={MaxEnt},
    description={Maximum Entropy Modeling},
    first={Clasificador de máxima entropia (MaxEnt por su nombre en inglés)\glsadd{MaxEntg}},
    see=[Glosario:]{MaxEntg}
}

\newglossaryentry{MaxEntg}{
    name={MaxEnt},
    description={Maximum Entropy Modeling como es conocido en inglés, es una técnica de aprendizaje máquina que consiste en encontrar las reglas de unos hechos, procurando que su probabilidad sea uniforme.}
}

\newglossaryentry{stopwords}{
    name={stopwords},
    description={Conjunto de palabras que se deben ignorar, de manera que no se filtre información que no es relevante para la tarea},
    first={Palabras detenidas o stopwords (cómo normalmente se les conoce)}
}

\newglossaryentry{NBg}{
    name=NB,
    description={Clasificador que aplica el teorema de Bayes, asumiendo que los hechos entre ellos son independientes}
}

\newglossaryentry{API}{
    type=\acronymtype,
    name={API},
    description={Interfaz de programación de aplicaciones},
    first={Interfaz de programación de aplicaciones (API por sus siglas en inglés)},
}

\newglossaryentry{SVM}{
    type=\acronymtype,
    name={SVM},
    description={Support Vector Machine},
    first={Maquina de soporte vectorial (SVM por sus siglas en inglés) \glsadd{SVMg}},
    see=[Glosario:]{SVMg}
}
\newglossaryentry{POS}{
    type=\acronymtype,
    name={POS},
    description={POS (Categoría gramatical, del inglés Part Of Speech)},
    first={Partes del habla (POS por sus siglas en inglés)\glsadd{POSg}},
    see=[Glosario:]{POSg}
}

\newglossaryentry{POSg}{
    name=POS,
    description={\textit{Part of speech} es la clase a la cual pertenecen las palabras ej. verbo, adverbio, sustantivo,etc.}
}

\newglossaryentry{SVMg}{
    name=SVM,
    description={Método de clasificación que maximiza la distancia entre las clases}
}

\newglossaryentry{embedding}{
    name=embedding,
    description={Encaje en español, es un concepto matemático que corresponde a una estancia de una estructura matemática la cual es contenida por otra instancia, en nuestro problema es la correspondencia entre dos objetos de diferentes espacios, por ejemplo el índice de una palabra y un vector que califica esa palabra según sus características: bueno, malo, etc. }
}


\newglossaryentry{bi-lstm}{
    type=\acronymtype,
    name={BI-LSTM},
    description={Redes bidireccionales de gran memoria de corto plazo},
    first={Redes bidireccionales de gran memoria de corto plazo (BI-LSTM por sus siglas en inglés)\glsadd{bi-lstmg}},
    see=[Glosario:]{bi-lstmg}
}

\newglossaryentry{bi-lstmg}{
    name=bi-lstm,
    description={Es una arquitectura de red neuronal que pertenece a la clasificación de recurrentes ya que pueden recordar lo que ya han visto, la bi-lstm consiste en una red neuronal que recuerda hacia adelante y puede olvidar dependiendo de su entrada actual y al resumen que se tiene de lo visto}
}

\newglossaryentry{mlp}{
    type=\acronymtype,
    name={MLP},
    description={Perceptrón multi capa.},
    first={Perceptrón multi capa (MLP por sus siglas en inglés) \glsadd{mlpg}},
    see=[Glosario:]{mlpg}
}

\newglossaryentry{mlpg}{
    name={mlp},
    description={Es uno de los nombres que se le da una red neuronal completamente conectada}
}

\newglossaryentry{backpropagation}{
    name={backpropagation},
    description={Es un método para entrenar redes neuronales que consiste en retroalimentar la red neuronal a la inversa, en lugar de predecir, se modifican los pesos de la red para obtener el resultado que minimice la función de costo}
}

\newglossaryentry{ReLU}{
    name={Rectified Linear Unit (ReLU)},
    description={Es una función que se emplea a la salida de una neurona, su formula generalmente es $f(x) = max(0,x)$}
}

\newglossaryentry{SGD}{
    type=\acronymtype,
    name={SGD},
    description={Stochastic Gradient Descent},
    first={Stochastic Gradient Descent(SGD)\glsadd{SGDg}},
    see=[Glosario:]{SGDg}
}

\newglossaryentry{SGDg}{
    name={SGDg},
    description={Algoritmo de optimización similar a gradient descent, pero que actúa con cada error que se computa}
}

\newglossaryentry{lstm}{
    type=\acronymtype,
    name={LSTM},
    description={Redes de gran memoria de corto plazo},
    first={Redes de gran memoria de corto plazo(LSTM por sus siglas en inglés)\glsadd{lstmg}},
    see=[Glosario:]{lstmg}
}

\newglossaryentry{lstmg}{
    name={LSTM},
    description={Arquitectura de redes neuronales recurrentes que tienen la capacidad de retener información de lo que ya han visto}
}

\newglossaryentry{precision}{
    type=\acronymtype,
    name={precisión},
    description={Métrica de desempeño\glsadd{precisiong}},
    first={precisión},
    see=[Glosario:]{precisiong}
}

\newglossaryentry{precisiong}{
    name={precisión},
    description={Es una métrica que consiste en medir qué tan bien funciona el modelo en detectar las muestras relevantes}
}

\newglossaryentry{exactitud}{
    type=\acronymtype,
    name={exactitud},
    description={Métrica de desempeño\glsadd{exactitudg}},
    see=[Glosario:]{exactitudg}
}

\newglossaryentry{exactitudg}{
    name={exactitud},
    description={Es una métrica que consiste en medir qué tantas veces se acierta en predecir una muestra cual sea}
}


\newglossaryentry{reclamo}{
    type=\acronymtype,
    name={reclamo},
    description={Métrica de desempeño\glsadd{reclamog}},
    see=[Glosario:]{reclamog}
}

\newglossaryentry{reclamog}{
    name={reclamo},
    description={Es una métrica que consiste en medir que porcentaje de las muestras positivas se logran encontrar}
}


\newglossaryentry{valor-F}{
    type=\acronymtype,
    name={valor-F},
    description={Métrica de desempeño\glsadd{valor-Fg}},
    see=[Glosario:]{valor-Fg}
}

\newglossaryentry{valor-Fg}{
    name={valor-F},
    description={Es una promedio ponderado qué tiene como propósito no propiciar que se alcancen altos puntajes si existen puntajes muy bajos, esto es para encontrar un mejor equilibrio, y no sólo un comportamiento promedio}
}


\newglossaryentry{padding}{
    name={padding},
    description={Es el proceso por el cuál un vector de cierta naturaleza se normaliza a una longitud específica, rellenando o truncando el vector original, en nuestro caso se lleno con 0 ya que el vector era numérico}
}