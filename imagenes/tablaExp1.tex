\begin{table}[h]
    \centering
    \caption{Tabla de métricas experimento 1, análisis por palabra. Nótese que la métrica con valor más alto es la exactitud, con un promedio de 0.9687; sin embargo, el reclamo cae debido a que no le es posible encontrar más muestras positivas, esto puede deberse a que ignora mucha información valiosa que contienen las palabras que se omiten. La desviación estándar resulta ser muy baja, lo cuál indica que este experimento no varía tanto y por lo tanto los promedios describen el comportamiento esencial del modelo.}
\begin{tabular}{|l|llll|}
\hline
              & Exactitud &     Precisión &     Reclamo  &   Valor-F \\ \hline

Subdivisión1            &       0.9698  &       0.9277  &       0.7589  &       0.8349  \\ 
Subdivisión2            &       0.9671  &       0.8422  &       0.8246  &       0.8333  \\ 
Subdivisión3            &       0.9692  &       0.9045  &       0.7720  &       0.8330  \\ 
Subdivisión4            &       0.9677  &       0.9569  &       0.7097  &       0.8150  \\ 
Subdivisión5            &       0.9698  &       0.9146  &       0.7694  &       0.8358  \\ \hline
Promedio                &       0.9687  &       0.9092  &       0.7669  &       0.8304  \\ \hline
Desviación estándar     &       0.0011  &       0.0378  &       0.0366  &       0.0078  \\ \hline

% Subdivisión 1 & 0.9709  &       0.9674  &       0.7226  &       0.8273 \\
% Subdivisión 2 & 0.9677  &       0.9529  &       0.7037  &       0.8095   \\
% Subdivisión 3 & 0.9647  &       0.9711  &       0.6793  &       0.7994   \\
% Subdivisión 4 & 0.9698  &       0.9719  &       0.7333  &       0.8359   \\
% Subdivisión 5 & 0.9722  &       0.9312  &       0.7713  &       0.8437  \\ \hline
% Promedio      & 0.9690  &       0.9589  &       0.7220  &       0.8232\\ \hline
\end{tabular}
		     \label{tab:exp1}
\end{table}