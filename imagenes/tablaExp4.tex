\begin{table}[h]
	\centering
	\caption{Tabla de métricas experimento 4, por tuplas de 3 caracteres. En esta tabla se puede ver una disminución considerable del valor-F, pero un aumente en el reclamo, además de una disminución en la desviación estándar, lo cual indica que es más estable, en cuando a reclamo. La precisión cae abruptamente y aumenta considerablemente su desviación estándar.}
\begin{tabular}{|l|llll|}
\hline
& Exactitud &     Precisión &     Reclamo  &   Valor-F \\ \hline
              
Subdivisión1            &       0.9754  &       0.8687  &       0.8895  &       0.8790  \\ 
Subdivisión2            &       0.9294  &       0.5959  &       0.9110  &       0.7205  \\ 
Subdivisión3            &       0.8768  &       0.4436  &       0.9409  &       0.6029  \\ 
Subdivisión4            &       0.9364  &       0.6259  &       0.9080  &       0.7410  \\ 
Subdivisión5            &       0.9507  &       0.7014  &       0.8837  &       0.7821  \\ \hline
Promedio                &       0.9338  &       0.6471  &       0.9066  &       0.7451  \\ \hline
Desviación estándar     &       0.0325  &       0.1389  &       0.0200  &       0.0896  \\ \hline



% Subdivisión 1 & 0.9800 & 0.9450 & 0.8479 & 0.8938  \\
% Subdivisión 2 & 0.9787 & 0.9340 & 0.8457 & 0.8877  \\
% Subdivisión 3 & 0.9788 & 0.9414 & 0.8443 & 0.8902  \\
% Subdivisión 4 & 0.9720 & 0.8247 & 0.9093 & 0.8650  \\
% Subdivisión 5 & 0.9743 & 0.8452 & 0.9116 & 0.8771  \\ \hline
% Promedio      & 0.9768 & 0.8718 & 0.8981 & 0.8828  \\ \hline
\end{tabular}
		     \label{tab:exp4}
\end{table}