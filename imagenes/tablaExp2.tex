\begin{table}[h]
    \centering
    \caption{Tabla de métricas experimento 2, análisis por carácter. Se nota un mejor desempeño respecto al primer experimento, debido a que las palabras que fueron ignoradas antes, ahora aportan su información. La mejor métrica de este experimento fue la extactitud, pero se nota una mejora significativa en la precisión y en el reclamo, de 5 y 10 puntos porcentuales respectivamente. La desviación estándar resulta ser muy baja, lo cuál indica que este experimento no varía tanto y por lo tanto los promedios describen el comportamiento esencial del modelo. }
\begin{tabular}{|l|llll|}
\hline
                        & Exactitud &     Precisión &     Reclamo  &   Valor-F \\ \hline
              
Subdivisión1            &       0.9840  &       0.9450  &       0.8928  &       0.9181  \\ 
Subdivisión2            &       0.9842  &       0.9646  &       0.8737  &       0.9169  \\ 
Subdivisión3            &       0.9811  &       0.9477  &       0.8568  &       0.8999  \\ 
Subdivisión4            &       0.9801  &       0.9562  &       0.8402  &       0.8944  \\ 
Subdivisión5            &       0.9833  &       0.9623  &       0.8674  &       0.9124  \\ \hline
Promedio                &       0.9825  &       0.9552  &       0.8662  &       0.9084  \\ \hline
Desviación estándar     &       0.0016  &       0.0078  &       0.0175  &       0.0095  \\ \hline
              
% Subdivisión 1 & 0.9716  &       0.8911  &       0.8130  &       0.8503 \\
% Subdivisión 2 & 0.9639  &       0.8648  &       0.7557  &       0.8066   \\
% Subdivisión 3 & 0.9768  &       0.8905  &       0.8808  &       0.8857   \\
% Subdivisión 4 & 0.9660  &       0.8683  &       0.7723  &       0.8175   \\
% Subdivisión 5 & 0.9707  &       0.8894  &       0.8096  &       0.8476   \\ \hline
% Promedio      & 0.9698  &       0.8808  &       0.8063  &       0.8415\\ \hline
\end{tabular}
		     \label{tab:exp2}
\end{table}