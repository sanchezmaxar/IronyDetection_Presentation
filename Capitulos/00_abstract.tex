\chapter*{}
\vspace*{-1.5cm}
{\let\clearpage\relax\section*{Resumen}}

\vspace*{-0.75cm}
\onehalfspacing
    \selectlanguage{spanish}
    \par La ironía es una recurso lingüístico que consiste en comunicar una intención contraria al significado literal de las palabras que se usan. Así mismo, se usa de manera constante en la vida cotidiana, por lo que tiene un gran valor para intereses particulares como son: mercadotecnia, evaluaciones de aceptación, opiniones sobre productos o servicios, entre otros. Este estudio propone un modelo de inteligencia artificial que soluciona la detección de la ironía en textos provenientes de la red social Twitter. Este estudio constituye cuatro experimentos, que difieren entre ellos por su preprocesamiento, y son los siguientes: análisis por palabra, análisis por carácter, análisis por tuplas de dos caracteres y análisis por tuplas de tres caracteres. El modelo consiste en de una arquitectura de red neuronal usando Redes bidireccionales de gran memoria de corto plazo, comunmente conocidas como BI-LSTM, que pasán su análisis a una red complemente conectada con una única salida, cuyo valor cuando es más cercano a 1 es más irónica y cuando es más cercana a 0 es menos irónica. Los resultados más relevantes fueron los encontrados en el tercer experimento, cuando se analizó por tuplas de 3 caracteres, cuya métrica valor-F fué en promedio 0.9157. El corpus que se usó fue el propuesto, generado y etiquetado por \textcite{lopez2016character}, que consiste en 76,530 tweets de los cuales 7,653 se etiquetaron como irónicos y 68,877 como no irónicos. El etiquetado de este corpus fue manual. Este trabajo fue supervisado por el Dr. Ivan Vladimir Meza Ruiz. Para consultar el código en \LaTeX y otras versiones, se puede seguir la siguiente liga: \url{https://github.com/sanchezmaxar/IronyDetection_Thesis}

    \vspace*{-1cm}
{\let\clearpage\relax\section*{Abstract}}
\onehalfspacing
\vspace*{-0.75cm}

\selectlanguage{english}
    \par Irony is a linguistic resource that consists in transmitting the opposite intention from the literal meaning of words that are used. Likewise, irony it's been used constantly on a regular basis; that is why it has a great value for particular interests just like: marketing, acceptance evaluations, opinions on products or services, among others. This study proposes a model of artificial intelligence that solves the detection of irony in the texts of the social network Twitter. This study contains four experiments, which differ among them by their preprocessing, and are the following: analysis by word, analysis by character, analysis by tuples of two characters and analysis by tuples of three characters. The model consists of a neural network architecture using bi-directional long short-term memory, commonly known as BI-LSTM, followed by a fully connected neural network with a single output, whose value when it is closer to 1 is more ironic and when it is closer 0 is less ironic. The most relevant results were the results from the third experiment, when analyzed by tuples of 3 characters, whose F-score metric was on average 0.9157. The corpus that was used was the one proposed, generated and labeled by \textcite{lopez2016character}, consisting of 76,530 tweets of which 7,653 were labeled as ironic and 68,877 as non-ironic. This study were supervised by the PhD. Iván Vladimir Meza Ruiz. For consulting \LaTeX code and different versions, you can check the following link: \url{https://github.com/sanchezmaxar/IronyDetection_Thesis}
    
    

